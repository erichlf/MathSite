\documentclass{hwset}

% info for header block in upper right hand corner
\name{Erich L Foster}
\class{Calculus I}
\duedate{Due: 07 February 2011}
\assignment{Homework 3}

\newcommand{\mtiny}[1]{\mbox{\tiny $#1$}}

\begin{document}
\begin{problem}[1.]
	\be
		\item \S 2.4 : $49$.
		\item \S 2.4 : $50$.
		\item \S 2.4 : $74$.
	\ee
\end{problem}

\be
	\item 
	\begin{solution}
		\begin{align*}
			\lim_{x\to -\infty} \frac{e^x - e^{-x}}{e^x + e^{-x}} &= \lim_{x\to
				-\infty} \frac{e^{x}\left(e^x - e^{-x}\right)}{e^x\left(e^x +
				e^{-x}\right)} \\
			&= \lim_{x\to -\infty} \frac{e^{2x} - 1}{e^{2x} + 1} \\
			&= \frac{0 - 1}{0 + 1} \\
			&= \boxed{-1}.
		\end{align*}
	\end{solution}
	\item 
	\begin{solution}
		\begin{align*}
			\lim_{x\to \infty} \dfrac{3x^2 + e^{-x}}{\sin \frac{1}{x} - 2x^2} &=
				\lim_{x\to \infty} \dfrac{\frac{1}{x^2}\left(3x^2 +
				e^{-x}\right)}{\frac{1}{x^2}\left(\sin \frac{1}{x} - 2x^2\right)} \\
			&=\lim_{x\to \infty} \dfrac{3 + \frac{e^{-x}}{x^2}}{\dfrac{\sin
				\frac{1}{x}}{x^2} - 2} \\
			&=\frac{3 + 0}{0 - 2} \\
			&=\boxed{-\frac{3}{2}}
		\end{align*}
	\end{solution}
	\item 
	\begin{solution}
		\begin{align*}
			\lim_{x\to \infty} \sqrt{x^2 + x} - \sqrt{x^2 - x} &= \lim_{x\to \infty}
				\left(\sqrt{x^2 + x} - \sqrt{x^2 - x}\right) \frac{\sqrt{x^2 + x} +
				\sqrt{x^2 - x}}{\sqrt{x^2 + x} + \sqrt{x^2 - x}} \\
			&= \lim_{x\to \infty} \frac{x^2 + x - x^2 + x}{\sqrt{x^2 + x} + \sqrt{x^2 - x}} \\
			&= \lim_{x\to \infty} \frac{2x}{\sqrt{x^2 + x} + \sqrt{x^2 - x}} \\
			&= \lim_{x\to \infty} \dfrac{\frac{1}{x} 2x}{\frac{1}{x}\left(\sqrt{x^2 + x}
				+ \sqrt{x^2 - x}\right)} \\
			&= \lim_{x\to \infty} \dfrac{2}{\sqrt{1 + \frac{1}{x}} + \sqrt{1 -
				\frac{1}{x}}} \\
			&= \dfrac{2}{\sqrt{1 + 0} + \sqrt{1 - 0}} \\
			&= \boxed{1}
		\end{align*}
	\end{solution}
\ee

\begin{problem}[2.] \S 2.5 : $51$.
\end{problem}
\be
	\item
	\begin{solution}
		We say that $f(x)$ approaches \tbf{infinity} as $x$ approaches $x_0$ from the
		\tbf{left}, and write
		\begin{equation*}
			\lim_{x\to x_0^-} f(x) = \infty,
		\end{equation*}
		if, for every positive real number B, there exists a corresponding number
		$\delta>0$ such that for all $x$
		\begin{equation*}
			x_0 - \delta < x < x_0 \quad \Rightarrow \quad f(x) > B.
		\end{equation*}
	\end{solution}
	\item
	\begin{solution}
		We say that $f(x)$ approaches \tbf{negative infinity} as $x$ approaches
		$x_0$ from the right, and write
		\begin{equation*}
			\lim_{x\to x_0^+} f(x) = -\infty,
		\end{equation*}
		if, for every \tbf{negative} real number B, there exists a corresponding
		number $\delta>0$ such that for all $x$
		\begin{equation*}
			x_0 < x < x_0 + \delta \quad \Rightarrow \quad f(x) < B.
		\end{equation*}
	\end{solution}
	\item
	\begin{solution}
		We say that $f(x)$ approaches \tbf{negative infinity} as $x$ approaches
		$x_0$ from the \tbf{left}, and write
		\begin{equation*}
			\lim_{x\to x_0^-} f(x) = -\infty,
		\end{equation*}
		if, for every \tbf{negative} real number B, there exists a corresponding
		number $\delta>0$ such that for all $x$
		\begin{equation*}
			x_0 - \delta < x < x_0 \quad \Rightarrow \quad f(x) < B.
		\end{equation*}
	\end{solution}
\ee

\begin{problem}[3.]
	Find any and all vertical and horizontal asymptotes of the following
	functions. Additionally, be sure to describe the behavior at the vertical
	asymptotes, i.e. evaluate the limits at those asymptotes. 
	\be
		\item $\dfrac{x^2-3x+2}{x^2-2x}$
		\item $\dfrac{3}{2} \left(\dfrac{x}{x-1}\right)^{2/3}$
	\ee
\end{problem}

\be
	\item
	\begin{solution}
		\tbf{Horizontal Asymptotes:}
		\begin{align*}
			\lim_{x\to \infty} \dfrac{x^2-3x+2}{x^2-2x} &= \lim_{x\to \infty}
				\dfrac{\frac{1}{x^2}\left(x^2-3x+2\right)}{\frac{1}{x^2}\left(x^2-2x\right)} \\
			&= \lim_{x\to \infty} \dfrac{1 - \frac{3}{x} + \frac{2}{x^2}}{1 -
				\frac{2}{x}} \\
			&= 1 \\
			\lim_{x\to -\infty} \dfrac{x^2-3x+2}{x^2-2x} &= \lim_{x\to -\infty}
				\dfrac{\frac{1}{x^2}\left(x^2-3x+2\right)}{\frac{1}{x^2}\left(x^2-2x\right)} \\
			&= \lim_{x\to -\infty} \dfrac{1 - \frac{3}{x} + \frac{2}{x^2}}{1 -
				\frac{2}{x}} \\
			&= 1
		\end{align*}
		Therefore the horizontal asymptote is $\boxed{y=1}$. \\
		\tbf{Vertical Asymptotes:}
		We are looking for $x$ values which result in the function becoming
		undefined. In this case we are looking for $x$ values that result in
		division by zero.
		\begin{align*}
			\dfrac{x^2-3x+2}{x^2-2x} &= \dfrac{\cancel{(x-2)}(x-1)}{\cancel{(x-2)}x} \\
			&= \dfrac{x-1}{x} \quad \text{for } x\ne 2 \\
			\Rightarrow \lim_{x\to 0^-} \dfrac{x^2-3x+2}{x^2-2x} &= \lim_{x\to 0^-}
				\dfrac{\begin{matrix}\mtiny{(-)} \\[-0.3em]
				x-1\end{matrix}}{\begin{matrix}x \\[-0.6em] \mtiny{(-)}\end{matrix}} \\
			&= \infty \\
			\Rightarrow \lim_{x\to 0^+} \dfrac{x^2-3x+2}{x^2-2x} &= \lim_{x\to 0^+}
				\dfrac{\begin{matrix}\mtiny{(-)} \\[-0.3em]
				x-1\end{matrix}}{\begin{matrix}x \\[-0.6em] \mtiny{(+)}\end{matrix}} \\
			&= -\infty 
		\end{align*}
		Therefore, the vertical asymptote is $\boxed{y=0}$.
	\end{solution}
	\item
	\begin{solution}
		\tbf{Horizontal Asymptotes:}
		\begin{align*}
			\lim_{x\to \infty} \dfrac{3}{2} \left(\dfrac{x}{x-1}\right)^{2/3} &=
				\dfrac{3}{2} \left(\lim_{x\to \infty} \dfrac{\frac{1}{x}(x)}{
				\frac{1}{x} (x-1)}\right)^{2/3} \\
			&= \dfrac{3}{2}\left(\lim_{x\to \infty} \dfrac{1}{1 -
				\frac{1}{x}}\right)^{2/3} \\
			&= \dfrac{3}{2}(1)^{2/3} \\
			&= \dfrac{3}{2} \\
			\lim_{x\to -\infty} \dfrac{3}{2} \left(\dfrac{x}{x-1}\right)^{2/3} &=
				\dfrac{3}{2} \left(\lim_{x\to -\infty} \dfrac{\frac{1}{x}(x)}{
				\frac{1}{x} (x-1)}\right)^{2/3} \\
			&= \dfrac{3}{2}\left(\lim_{x\to -\infty} \dfrac{1}{1 -
				\frac{1}{x}}\right)^{2/3} \\
			&= \dfrac{3}{2}(1)^{2/3} \\
			&= \dfrac{3}{2}
		\end{align*}
		Therefore the horizontal asymptote is $\boxed{y=\dfrac{3}{2}}$. \\
		\tbf{Vertical Asymptotes:}
		We are looking for $x$ values which result in the function becoming
		undefined. In this case we are looking for $x$ values that result in
		division by zero.
		\begin{align*}
			\Rightarrow \lim_{x\to 1^-} \dfrac{3}{2}\left(\dfrac{x}{x-1}\right)^{2/3}
				&= \dfrac{3}{2}\left(\lim_{x\to 1^-} \dfrac{\begin{matrix}\mtiny{(+)} \\[-0.3em]
				x\end{matrix}}{\begin{matrix}x-1 \\[-0.6em]
				\mtiny{(-)}\end{matrix}}\right)^{2/3} \\
			&= \dfrac{3}{2}(-\infty)^{2/3} \\
			&= \infty \\
			\Rightarrow \lim_{x\to 1^+} \dfrac{3}{2}\left(\dfrac{x}{x-1}\right)^{2/3}
				&= \dfrac{3}{2}\left(\lim_{x\to 1^+} \dfrac{\begin{matrix}\mtiny{(+)} \\[-0.3em]
				x\end{matrix}}{\begin{matrix}x-1 \\[-0.6em]
				\mtiny{(+)}\end{matrix}}\right)^{2/3} \\
			&= \dfrac{3}{2}(\infty)^{2/3} \\
			&= \infty 
		\end{align*}
		Therefore, the vertical asymptote is $\boxed{y=1}$.
	\end{solution}
\ee

\begin{problem}[4.] For what value(s) of $b$ is the function $g$ continuous
on $(-\infty,\infty)$?
	\begin{equation*}
		g(x)=\begin{cases} 
			-4x+1 & x\leq b \\
				x^2-2x+2 & x>b
			\end{cases}
	\end{equation*}
\end{problem}

\begin{solution}
	For this function to be continuous we want the limit to exist and we want
	\begin{equation*}
		\lim_{x\to b} f(x) = f(b).
	\end{equation*}
	Let's start with the existence of the limit. For a limit to exist we want
	\begin{align*}
		\lim_{x\to b^-} f(x) &= \lim_{x\to b^+} f(x) \\
		\lim_{x\to b^-} -4x+1 &= \lim_{x\to b^+} x^2-2x+2 \\
		-4b+1 &= b^2-2b+2 \\
		b^2+2b+1 &= 0 \Rightarrow \boxed{b=-1}\\
	\end{align*}
	Since, $-4x +1 = x^2-2x +2$ at $x=b=-1$ we also know that the limit is equal to
	the function value at $x=b=-1$. Therefore, the function is continuous when
	$b=-1$.
\end{solution}

\end{document}
