\documentclass{hwset}

% info for header block in upper right hand corner
\name{Erich L Foster}
\class{Calculus I}
\duedate{Due: 15 February 2011}
\assignment{Homework 4}

\begin{document}
\begin{problem}[1.]
	Use the Intermediate Value Theorem to show that there exist a root of the
	given equation in the specified interval.
	\be
		\item $\cos(x)=x\, \text{ for } x\in (0,1)$
		\item $\sqrt[3]{x}=1-x\, \text{ for } x\in(0,1)$
	\ee
\end{problem}

\be
	\item
	\begin{solution}
		Let $f(x) = \cos x - x$ since $\cos x$ and $x$ are both continuous, $f(x)$ is
		continuous.
		\begin{align*}
			f(0) &= \cos 0 - 0 = 1 \\
			f(1) &= \cos 1 - 1 \approx -0.46
		\end{align*}
		Thus, since $f(x)$ changes sign in the interval $(0,1)$ and is continuous in
		$(0,1)$, by the IVT there exists a $c\in(0,1)$ such that $f(c)=0$.
		Therefore, there exists an $x\in(0,1)$ such that $\cos x = x$. 
	\end{solution}
	\item
	\begin{solution}
		Let $f(x) = \sqrt[3]{x} + x - 1$ since $\sqrt[3]{x}$ and $x - 1$ are both
		continuous, $f(x)$ is continuous.
		\begin{align*}
			f(0) &= \sqrt[3]{0} + 0 - 1 = -1 \\
			f(1) &= \sqrt[3]{1} + 1 -1 = 1.
		\end{align*}
		Thus, since $f(x)$ changes sign in the interval $(0,1)$ and is continuous in
		$(0,1)$, by the IVT there exists a $c\in(0,1)$ such that $f(c)=0$.
		Therefore, there exists an $x\in (0,1)$ such that $\sqrt[3]{x} = 1 - x$. 
	\end{solution}
\ee

\begin{problem}[2.]
	Is there a real number that is exactly $1$ more that its cube? Why?
\end{problem}

\begin{solution}
	The question is asking if there is an $x$ such that $x^3 + 1 = x$. Let $f(x) =
	x^3 - x + 1$, which is continuous since it is a polynomial. Looking at the
	interval $[-2,1]$ we see that 
	\begin{align*}
		f(-2) &= -2^3 + 2 + 1 = -5 \\
		f(0) &= 0^3 - 0 + 1 = 1.
	\end{align*}
		Thus, since $f(x)$ changes sign in the interval $[-2,0]$ and is continuous
		in $[-2,0]$, by the IVT there exists a $c\in[-2,0]$ such that $f(c)=0$.
		Therefore, there exists an $x\in[-2,0]$ such that $x^3 + 1 = x$. 
\end{solution}

\begin{problem}[3.] 
	Find the slope of the tangent line to 
	\begin{equation*}
		f(x) = \sqrt{x}
	\end{equation*}
	at the point $x_0 = 1$ and then determine the equation of the tangent line.
\end{problem}

\begin{solution}
	Using the definition of the slope of a curve at a point we see
	\begin{align*}
		m &= \lim_{h\to 0} \dfrac{\sqrt{x_0 + h} - \sqrt{x_0}}{h} \\
		&= \lim_{h\to 0} \dfrac{\sqrt{x_0 + h} - \sqrt{x_0}}{h}
			\dfrac{\sqrt{x_0+h}+\sqrt{x_0}}{\sqrt{x_0+h}+\sqrt{x_0}} \\
		&= \lim_{h\to 0} \dfrac{x_0 + h - x_0}{h
			\left(\sqrt{x_0+h}+\sqrt{x_0}\right)} \\
		&= \lim_{h\to 0} \dfrac{1}{\sqrt{x_0+h}+\sqrt{x_0}} \\
		&= \dfrac{1}{2\sqrt{x_0}} \\
		&= \dfrac{1}{2\sqrt{1}} \\
		&= \boxed{\dfrac{1}{2}.}
	\end{align*}
	Now using the point-slope form at the point $(1,1)$ we have
	\begin{align*}
		y - 1 &= \frac{1}{2}(x - 1) \\
		y &= \frac{1}{2}x - \frac{1}{2} + 1 \\
		\boxed{y = \frac{1}{2}x + \frac{1}{2}.}
	\end{align*}	
\end{solution}

\begin{problem}[4.] 
	Using the definition of the derivative find the derivative of 
	\begin{equation*}
		f(x) = \sin x
	\end{equation*}
	Hint: Use the relation $\sin x - \sin y =	2 \cos\left(\frac{x +
			y}{2}\right)\cdot \sin\left( \frac{x - y}{2}\right)$  
\end{problem}

\begin{solution}
	Using the definition of the derivative we have
	\begin{align*}
		f'(x) &= \lim_{h\to 0} \dfrac{\sin (x + h) - \sin x}{h} \\ 
		&= \lim_{h\to 0} \dfrac{2\cos\left(\frac{2x + h}{2}\right)\cdot \sin
			\left(\frac{h}{2}\right)}{h} \\ 
		&= \lim_{h\to 0} \cos\left(\frac{2x + h}{2}\right) \cdot \cancelto{1}{\lim_{h\to 0}
			\dfrac{\sin \left(\frac{h}{2}\right)}{\frac{h}{2}}} \\ 
		&= \boxed{\cos x.} 
	\end{align*}
\end{solution}

\begin{problem}[5.] 
	Suppose $f(x)$ is given by
	\begin{equation*}
		f(x)=\begin{cases}
			x^{3}+x-1 & x< 0 \\
			x^{2}-1 &		0\leq x\leq 1 \\
			x^{3}-3x^{2}+3x & x > 1
		\end{cases}.
	\end{equation*}
	\be
		\item Does $f'(0)$ exist? If so, what is it? If not, why not? Explain!\\
		\item Does $f'(1)$ exist? If so, what is it? If not, why not? Explain!\\
	\ee
\end{problem}

\be
	\item
	\begin{solution}
		For the derivative to exist at $x=0$ we must check if the right hand and
		left hand derivatives are equal at the point $x=0$. Thus, the left hand
		derivative is
		\begin{align*}
			\lim_{h\to 0^-} \dfrac{f(x_0+h) - f(x_0)}{h} &= \lim_{h\to 0^-}
				\lim_{h\to 0^-} \dfrac{f(0+h) - f(0)}{h} \\
			&= \lim_{h\to 0^-} \dfrac{h^3 + h - 1 - 0^2 + 1}{h} \\
			&= \lim_{h\to 0^-} \dfrac{h^3 + h}{h} \\
			&= \lim_{h\to 0^-} h^2 + 1 \\
			&= 1 
		\end{align*}
		and the right hand derivative is
		\begin{align*}
			\lim_{h\to 0^+} \dfrac{f(x_0+h) - f(x_0)}{h} &= 
				\lim_{h\to 0^+} \dfrac{f(0+h) - f(0)}{h} \\
			&=\lim_{h\to 0^+} \dfrac{(0+h)^2 - 1 - 0^2 + 1}{h} \\
			&= \lim_{h\to 0^+} \dfrac{h^2}{h} \\
			&= \lim_{h\to 0^+} h \\
			&= 0. 
		\end{align*}
		Therefore, the left and right hand derivatives are not equal at $x=0$ and so
		$\boxed{f'(0) \text{ does not exist.}}$
	\end{solution}
	\item
	\begin{solution}
		Again, for the derivative to exist at $x=1$ we must check if the right hand and
		left hand derivatives are equal at the point $x=1$. Thus, the left hand
		derivative is
		\begin{align*}
			\lim_{h\to 0^-} \dfrac{f(x_0+h) - f(x_0)}{h} &=
				\lim_{h\to 0^-} \dfrac{f(1+h) - f(1)}{h} \\ 
			&=\lim_{h\to 0^-} \dfrac{(1+h)^2 - 1 - 1^2 + 1}{h} \\
			&= \lim_{h\to 0^-} \dfrac{1 + 2h + h^2 - 1}{h} \\
			&= \lim_{h\to 0^-} \dfrac{2h + h^2}{h} \\
			&= \lim_{h\to 0^-} 2 + h \\
			&= 2 
		\end{align*}
		and the right hand derivative is
		\begin{align*}
			\lim_{h\to 0^+} \dfrac{f(x_0+h) - f(x_0)}{h} &= 
				\lim_{h\to 0^-} \dfrac{f(1+h) - f(1)}{h} \\ 
			&= \lim_{h\to 0^-} \dfrac{(1+h)^3 - 3(1 + h)^2 + 3(1 + h) - 1^2 + 1}{h} \\
			&= \lim_{h\to 0^+} \dfrac{1 + 3h + 3h^2 + h^3 - 3 - 6h - 3h^2 + 3 + 3h}{h} \\
			&= \lim_{h\to 0^+} \dfrac{1 + h^3}{h} \\
			&= \lim_{h\to 0^+} \dfrac{1}{h} + \lim_{h\to 0^+} h^2 \\
			&= +\infty 
		\end{align*}
		Therefore, the left and right hand derivatives are not equal at $x=0$ and,
		in fact the right hand derivative doesn't exist since it approaches
		$+\infty$ when we approach $x=1$ from the right . Therefore, $\boxed{f'(1) \text{ does not exist.}}$
	\end{solution}
\ee
\end{document}
