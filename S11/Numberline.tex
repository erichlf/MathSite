\documentclass{hwset}

\name{Erich L Foster}
\class{Calculus I}
\duedate{28 February 2011}
\assignment{Numberline}

\begin{document}
Say you are given a problem where you need to determine when $f'(t)<0$ given
$f'(t) = t^2+2t-3$ then the easiest way to do this problem is by a numberline.
First find the zeros of $f'(t)$, i.e.
\begin{align*}
	&f'(t) = (t+3)(t-2) = 0 \\
	&\Rightarrow t+3 = 0 \quad t - 2 = 0 \\
	&\Rightarrow t = -3 \quad t = 2
\end{align*}
Now plot a number line like so \\
\begin{tikzpicture}%[font=\Large]
	\draw (0,0) -- (12.5,0);
	\foreach \x in {0,3,6,9,12}
		\draw[shift={(\x,0)},color=black] (0pt,3pt) -- (0pt,0pt);
%	\draw (0,0) node[above]{$(+)$};
	\draw (3,0) node[above]{$-3$};
%	\draw (6,0) node[above]{$(-)$};
	\draw (9,0) node[above]{$2$};
%	\draw (12,0) node[above]{$(+)$};
	\draw (13,0) node[right]{$f'(t)$};
\end{tikzpicture} \\
Now, that you have a number line, determine the sign $f'(t)$ in each interval.
To do this evaluate $f'(t)$ at some point in each interval, i.e
\begin{align*}
	f'(-2) &= (-4+3)(-4-2) = 6 \\
	f'(0) &= (0 + 3)(0-2) = -6 \\ 
	f'(3) &= (3 + 3)(3-2) = 6. 
\end{align*}
Now, plot the number line with the positive and negatives in it \\
\begin{tikzpicture}%[font=\Large]
	\draw (0,0) -- (12.5,0);
	\foreach \x in {0,3,6,9,12}
		\draw[shift={(\x,0)},color=black] (0pt,3pt) -- (0pt,0pt);
	\draw (0,0) node[above]{$(+)$};
	\draw (3,0) node[above]{$-3$};
	\draw (6,0) node[above]{$(-)$};
	\draw (9,0) node[above]{$2$};
	\draw (12,0) node[above]{$(+)$};
	\draw (13,0) node[right]{$f'(t)$};
\end{tikzpicture}
From this we see that $f'(t)<0$ in the interval $t\in(-3,2)$ and $f'(t)>0$ in
the intervals $t\in(-\infty,3)\bigcup (2,\infty)$.

\end{document}
