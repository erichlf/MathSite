\documentclass{hwset}

\name{Erich L Foster}
\class{Calculus I}
\date{13 April 2011}
\assignment{Homework 10}

\begin{document}
\begin{problem}[1.]
	Use Linearization to approximate the following values. Additionally, find
	the error and the relative error.
	\begin{enumerate}
		\item $\sqrt[3]{1.1}$
		\item $\tan^{-1} (\sqrt{3} + 0.15)$ (Convert your answer to degrees)
	\end{enumerate}
\end{problem}

\begin{solution}
	To use a linearization we must first have an $x_0$ for which we know the exact
	value of $f(x_0)$.
	\begin{enumerate}
		\item Let $f(x) = \sqrt[3]{x}$ then the linearization of $f(x)$ about $x_0 =
			1$ is
			\begin{align*}
				L(x) &= f(x_0) + f'(x_0)(x-x_0) \\
				&= 1 + \frac{1}{3 (1)^{2/3}}(x-1) \\
				&= 1 + \frac{1}{3}(x-1) 
			\end{align*}
			Thus, the approximate value to $\sqrt[3]{1.1}$ is 
			\begin{equation*}
				L(1.1) \approx 1 + 0.0333 = \boxed{1.0333}
			\end{equation*}
			The error is 
			\begin{align*}
				\epsilon &= |f(1.1) - L(1.1)|\\
				&= |1.0323 - 1.0333| = \boxed{1\times 10^{3}}
			\end{align*}
			and the relative error is
			\begin{align*}
				\varepsilon &= \left|\frac{f(1.1) - L(1.1)}{f(1.1)}\right|\\
				&\approx \left|\frac{1.0323 - 1.0333}{1.0323}\right| \approx
				\boxed{1\times 10^{-3}}
			\end{align*}
		\item Let $f(x) = \tan^{-1} x$ then the linearization of $f(x)$ about $x_0 =
			\sqrt{3}$ is
			\begin{align*}
				L(x) &= f(x_0) + f'(x_0)(x-x_0) \\
				&= \frac{\pi}{3} + \frac{1}{1+(\sqrt{3})^2}(x-\sqrt{3}) \\
				&= \frac{\pi}{3} + \frac{1}{4}(x-\sqrt{3}) \\
			\end{align*}
			Thus, the approximate value of $\tan^{-1} (\sqrt{3} + 0.15)$ is 
			\begin{align*}
				L(\sqrt{3}+0.15) &= \frac{\pi}{3} + \frac{1}{4}(\sqrt{3} + 0.15 -
					\sqrt{3}) \\
				&= \frac{\pi}{3} + \frac{1}{4}  0.15 \\
				&\approx 1.085 = 1.085 \cdot \frac{180^\circ}{\pi} \approx
				\boxed{62.15^\circ}
			\end{align*}
			The error is 
			\begin{align*}
				\epsilon &= |f(\sqrt{3}+0.15) - L(\sqrt{3}+0.15)|\\
				&\approx |62.02^\circ - 62.15^\circ| = \boxed{0.13^\circ}
			\end{align*}
			and the relative error is
			\begin{align*}
				\varepsilon &= \left|\frac{f(1.1) - L(1.1)}{f(1.1)}\right|\\
				&\approx \left|\frac{0.13^\circ}{62.02^\circ}\right| \approx \boxed{2.1
				\times 10^{-3}}
			\end{align*}
	\end{enumerate}
\end{solution}

\begin{problem}[2.]
	Using Newton's Method find the root to the following functions, to the nearest
	hundreth. (You must show each iteration to get full credit)
	\begin{enumerate}
		\item $f(x) = x^5 + x + 1$
		\item $g(x) = \cos^{-1} x - e^x$
	\end{enumerate}
\end{problem}

\begin{enumerate}
	\item \begin{solution} Let $x_0=0$ and $'f(x) = 5x^4 + 1$ then we will use a
	table to find the solution
		\begin{equation*}
			\begin{array}{|c|c|c|c|c|}
				\hline
				n & x_n & f(x_n) & f'(x_n) & x_{n+1}=x_n-\frac{f(x_n)}{f'(x_n)} \\
				\hline
				0 & 0 & 1 & 1 & -1 \\
				1 & -1 & -1 & 6 & -0.833 \\
				2 & -0.833 & -0.235 & 3.411 & -0.764 \\
				3 & -0.764 & -0.025 & 2.707 & -0.755 \\
				4 & -0.755 & -0.000 & 2.625 & -0.755 \\
				\hline
			\end{array}
		\end{equation*}
		By the fifth iteration we see that difference between $x_5$ and $x_4$ is
		less than $0.01$ and so	our answer is $x\approx -0.755$.
	\end{solution}
	\item \begin{solution} Let $x_0=0$ and $'f(x) = -\frac{1}{\sqrt{1-x^2}} -
	e^x<++>$ then we will use a table to find the solution
		\begin{equation*}
			\begin{array}{|c|c|c|c|c|}
				\hline
				n & x_n & f(x_n) & f'(x_n) & x_{n+1}=x_n-\frac{f(x_n)}{f'(x_n)} \\
				\hline
				0 & 0.000 & 0.571 & -2.000 & 0.285 \\
				1 & 0.285 & -0.049 & -2.374 & 0.265 \\
				2 & 0.265 & 0.000 & -2.340 & 0.265 \\
				\hline
			\end{array}
		\end{equation*}
		By the fifth iteration we see that difference between $x_2$ and $x_3$ is
		less than $0.01$ and so	our answer is $x\approx 0.265$.
	\end{solution}
\end{enumerate}

\begin{problem}[3.]
	Using the Taylor series expansion about $\theta_0 = 0$ show that 
	\begin{equation*}
		e^{i \theta} = \cos \theta + i \sin \theta
	\end{equation*}
	where $i^2 = -1$. 
\end{problem}

\begin{solution}
	Let $g(\theta) = \cos \theta + i \sin \theta$ then
	\begin{equation*}
		\begin{array}{|c|c|c|}
			\hline
			n & g^{(n)}(\theta) & g^{(n)}(\theta_0) \\
			\hline
			0 & \cos \theta + i \sin \theta & 1 \\
			1 & -\sin \theta + i \cos \theta & i \\
			2 & -\cos \theta - i \sin \theta & -1 \\
			3 & \sin \theta - i \cos \theta & -i \\
			4 & \cos \theta + i \sin \theta & 1 \\
			\hline
		\end{array}
	\end{equation*}
	Thus, we see that the Taylor series is
	\begin{align*}
		g(\theta) &\approx g(\theta_0) + g'(\theta_0)(\theta - \theta_0) +
			\frac{g''(\theta_0)}{2!}(\theta - \theta_0) +
			\frac{g'''(\theta_0)}{3!}(\theta-\theta_0)^3 + \cdots \\
		&= 1 + i \theta - \frac{\theta^2}{2} - \frac{i\theta^3}{3!} +
			\frac{\theta^4}{4!} + \cdots
	\end{align*}
	On the other hand let $f(x) = e^{i\theta}$ then
	\begin{equation*}
		\begin{array}{|c|c|c|}
			\hline
			n & f^{(n)}(\theta) & f^{(n)}(\theta_0) \\
			\hline
			0 & e^{i\theta} & 1 \\
			1 & ie^{i\theta} & i \\
			2 & -e^{i\theta} & -1 \\
			3 & -ie^{i\theta} & -i \\
			4 & e^{i\theta} & 1 \\
			\hline
		\end{array}
	\end{equation*}
	Thus, we see that the Taylor series is
	\begin{align*}
		f(\theta) &\approx f(\theta_0) + f'(\theta_0)(\theta - \theta_0) +
			\frac{f''(\theta_0)}{2!}(\theta - \theta_0) +
			\frac{f'''(\theta_0)}{3!}(\theta-\theta_0)^3 + \cdots \\
		&= 1 + i \theta - \frac{\theta^2}{2} - \frac{i\theta^3}{3!} +
			\frac{\theta^4}{4!} + \cdots
	\end{align*}
	and we see that $e^{i\theta} = \cos \theta + i \sin \theta$.
\end{solution}

\begin{problem}[4.]
	Find the fourth-order Taylor series expansion of the following function about
	the given $x_0$.
	\begin{enumerate}
		\item $f(x) = 2x^4 + 3x^2 - x + 4,\; x_0 = 0$ (Simplify)
		\item $g(x) = \ln x,\; x_0 = 1$
	\end{enumerate}
\end{problem}

\begin{enumerate}
	\item \begin{solution}
			\begin{equation*}
				\begin{array}{|c|c|c|}
					\hline
					n & f^{(n)}(x) & f^{(n)}(x_0) \\
					\hline
					0 & 2x^4+3x^2-x+4 & 4 \\
					1 & 8x^3+6x-1 & -1 \\
					2 & 24x^2+6 & 6 \\
					3 & 48x & 0 \\
					4 & 48 & 48 \\
					\hline
				\end{array}
			\end{equation*}
			And so the fourth-order taylor series is
			\begin{align*}
				p_4(x) &= f(x_0) + f'(x_0)(x-x_0) + \frac{f''(x_0)}{2!}(x-x_0)^2 +
					\frac{f'''(x_0)}{3!}(x-x_0)^3 + \frac{f^{(4)}(x_0)}{4!}(x-x_0)^4 \\
				&= 4 - x + \frac{6}{2} x^2 + \frac{48}{24}x^4 \\
				&= \boxed{4 - x + 3 x^2 + 2 x^4}
			\end{align*}
		\end{solution}
	\item \begin{solution}
			\begin{equation*}
				\begin{array}{|c|c|c|}
					\hline
					n & g^{(n)}(x) & g^{(n)}(x_0) \\
					\hline
					0 & \ln x & 0 \\
					1 & \frac{1}{x} & 1 \\
					2 & -\frac{1}{x^2} & -1 \\
					3 & \frac{2}{x^3} & 2 \\
					4 & -\frac{6}{x^4} & -6 \\
					\hline
				\end{array}
			\end{equation*}
			And so the gourth-order taylor series is
			\begin{align*}
				p_4(x) &= g(x_0) + g'(x_0)(x-x_0) + \frac{g''(x_0)}{2!}(x-x_0)^2 +
					\frac{g'''(x_0)}{3!}(x-x_0)^3 + \frac{g^{(4)}(x_0)}{4!}(x-x_0)^4 \\
				&= \boxed{(x-1) - \frac{1}{2} (x-1)^2 + \frac{2}{3!}(x-1)^3 -
				\frac{6}{4!}(x-1)^4} 
			\end{align*}
		\end{solution}
\end{enumerate}

\begin{problem}[5.]
	Find the absolute maximum and absolute minimum values of $f$ on the given
	interval.
	\begin{enumerate}
		\item $f(x) = x^3 - 12x + 1,\; [-3,5]$
		\item $f(x) = x - 2\cos x,\; [-\pi, \pi]$
	\end{enumerate}
\end{problem}

\begin{enumerate}
	\item \begin{solution}
			First we must find all critical points (in the given interval)
			\begin{align*}
				f'(x) &= 3x^2 - 12 = 0\\
				\Rightarrow & x = \{\pm 2\}
			\end{align*}
			Now, we check the value of the function at the end points and the critical
			points
			\begin{align*}
				f(-3)&= 10 \\
				f(5)&= 66 \\
				f(2)&= -15 \\
				f(-2)&= 17 
			\end{align*}
			Therefore, we see that the \tbf{absolute max} is $\boxed{66 \text{ at } x=5}$ and
			the \tbf{absolute min} is $\boxed{-15 \text{ at } x=2.}$
		\end{solution}
	\item \begin{solution}
			First we must find all critical points (in the given interval)
			\begin{align*}
				f'(x) &= 1 + 2 \sin x = 0\\
				\Rightarrow & x = \left\{-\frac{\pi}{6},-\frac{5\pi}{6}\right\} 
			\end{align*}
			Now, we check the value of the function at the end points and the critical
			points
			\begin{align*}
				f(-\pi)&= -\pi + 2 \\
				f(\pi)&= \pi + 2\\
				f(\sfrac{-\pi}{6})&= \sfrac{-\pi}{6} -\sqrt{3} \\
				f(\sfrac{-5\pi}{6})&= \sfrac{-5\pi}{6} +\sqrt{3}.
			\end{align*}
			Therefore, we see that the \tbf{absolute max} is 
			$\boxed{\pi + 2 \text{ at	} x=\pi}$ 
			and the \tbf{absolute min} is 
			$\boxed{\sfrac{-\pi}{6} - \sqrt{3} \text{ at } x=\sfrac{-\pi}{6}.}$
		\end{solution}
\end{enumerate}

\end{document}
