\documentclass{hwset}

% info for header block in upper right hand corner
\name{Erich L Foster}
\class{Calculus I}
\duedate{Due: 02 March 2011}
\assignment{Homework 6}

\begin{document}

\begin{problem}[1.] 
	Find the derivative and second derivative of the following
	\be
		\item $g(x) = \frac{1+\sec(x)}{1-\sec(x)}$
		\item $h(x) = \sin^2 x$
	\ee
\end{problem}

\be
	\item
	\begin{solution}
		\begin{align*}
			g'(x) &= \dfrac{\frac{d}{dx}\left[1+\sec x \right](1-\sec x) -
				\frac{d}{dx}\left[1 - \sec x\right](1 + \sec x}{(1 - \sec x)^2} \\
			&= \dfrac{\sec x \tan x (1-\sec x) + \sec x \tan x(1 + \sec x)}{(1 - \sec
				x)^2} \\
			&= \boxed{\dfrac{2\sec x \tan x}{(1 - \sec x)^2}} \\
			g''(x)&= \dfrac{\frac{d}{dx}\left[2\sec x \tan x\right](1 - \sec x)^2 -
				\frac{d}{d(1- \sec x)}\left[(1 - \sec x)^2\right]
				\frac{d}{dx}\left[1-\sec x\right](2\sec x \tan x)}{(1-\sec x)^4} \\
			&= \dfrac{2\left(\frac{d}{dx}\left[\sec x\right]\tan x +
				\frac{d}{dx}\left[\tan x\right] \sec x\right)(1 - \sec x)^2 -
				2(1 - \sec x)(\sec x \tan x)(2\sec x \tan x)}{(1-\sec x)^4} \\
			&= \boxed{\dfrac{2\left(\sec x\tan^2 x + \sec^3 x\right)(1 - \sec x)^2 -
				2(1 - \sec x)(\sec x \tan x)(2\sec x \tan x)}{(1-\sec x)^4}}
		\end{align*}
	\end{solution}
	\item
	\begin{solution}
		\begin{align*}
			h'(x) &= \frac{d}{d(\sin x)}\left[(\sin x)^2\right] \frac{d}{dx}\left[\sin
				x\right] \\
			&= 2\sin x \cos x \\
			&= \boxed{\sin 2 x} \\
			h''(x) & = \frac{d}{d(2x)}\left[\sin 2x\right] \frac{d}{dx}(2x) \\
			&= (\cos 2x) 2 \\
			&= \boxed{2 \cos 2x}
		\end{align*}
	\end{solution}
\ee

\begin{problem}[2.]
	Suppose that an object moves back and forth according to the function
	\begin{equation*}
		f(t)=t^{3}+bt^{2}+ct+d,\quad f(0)=1,\, f'(0)=0, \text{ and } f''(0)=3.
	\end{equation*}
	\be
		\item Using the information above find $f(t)$.
		\item When is the object at rest?
		\item When is the object moving forward? Moving backward?
		\item When is the object accelerating?
		\item	How far did the object travel (counting retraces!) between $t=0$ and
		$t=8$?
	\ee
\end{problem}

\be
	\item
	\begin{solution}
		\begin{align*}
			&f(0) = d = 1 \\
			&f'(t) = 3t^2 + 2bt + c \\
			&f'(0) = c = 0 \\
			&f''(t) = 6t+2b \\
			&f''(0) = 2b = 3 \Rightarrow b = \frac{3}{2} \\
			\Rightarrow & \boxed{f(t) = t^3 + \frac{3}{2} t^2 + 1} 
		\end{align*}
	\end{solution}
	\item
	\begin{solution}
	The object is at rest when $f'(t)$ (velocity) is zero, i.e.
	\begin{align*}
		&f'(t) = 3t^2 + 3t = 0 \\
		&3t(t + 1) = 0\\
		&\boxed{t = 0, \quad t = -1}
	\end{align*}
	\end{solution}
	\item
	\begin{solution}
		An object is moving forward when $f'(t)>0$ and backward when $f'(t)<0$, so
		give the zeros we found in (b) we can draw the following number line \\
\begin{tikzpicture}%[font=\Large]
	\draw (0,0) -- (12.5,0);
	\foreach \x in {0,3,6,9,12}
		\draw[shift={(\x,0)},color=black] (0pt,3pt) -- (0pt,0pt);
%	\foreach \x in {0,1,2,3,4,5,6,7,8,9,10,11,12}
%		\draw[shift={(\x,0)},color=black] (0pt,0pt) -- (0pt,-3pt) node[below] 
%			{$\frac{\x}{3}$};
	\draw (0,0) node[above]{$(+)$};
	\draw (3,0) node[above]{$-1$};
	\draw (6,0) node[above]{$(-)$};
	\draw (9,0) node[above]{$0$};
	\draw (12,0) node[above]{$(+)$};
	\draw (13,0) node[right]{$f'(t)$};
\end{tikzpicture} \\
		From the above number line we see that the particle is moving forward when
		$\boxed{t\in(-\infty,-1)\bigcup(0,\infty)}$ and moving backward when
		$\boxed{t\in(-1,0)}$.
	\end{solution}
	\item
	\begin{solution}
		The object is accelerating when $f''(t)>0$, i.e.
		\begin{align*}
			f''(t) &= 6t+3 > 0 \\
			&6t > -3 \\
			&\boxed{t > -\frac{1}{2}}
		\end{align*}
		Note: On a more complicated problem a number line could be used.
	\end{solution}
	\item
	\begin{solution}
		Since the object is only moving forward in the given interval we can use the
		displacement formula
		\begin{align*}
			\Delta f &= f(8) - f(0) \\
			&= 8^3 + \frac{3}{2} 8^2 + 1 - 0^3 - \frac{3}{2} 0^2 - 1 \\
			&= 512 + \frac{3}{2}\cdot 64 \\
			&= \boxed{608}
		\end{align*}
	\end{solution}
\ee

\begin{problem}[3.] 
	Suppose $f(u)=\cos(u)$ and $g(t)=3t^{4}$. Using chain rule, compute:
	\be
		\item $(f\circ g)'(t)$
		\item $(g\circ f)'(u)$
		\item $(g\circ g)'(t)$
		\item $(f\circ f)'(u)$
	\ee
\end{problem}

\begin{enumerate}
	\item \begin{solution}
		\begin{align*}
			(f \circ g)'(t) &= f'(g(t)) g'(t)	\\
			&= \boxed{-\sin (3t^4) \cdot 12t^3}
		\end{align*}
	\end{solution}
	\item \begin{solution}
		\begin{align*}
			(g \circ f)'(u) &= g'(f(u)) f'(u) \\
			&= \boxed{-12 \cos^3 u \sin u}
		\end{align*}
	\end{solution}
	\item \begin{solution}
		\begin{align*}
			(g\circ g)'(t) &= g'(g(t)) g'(t) \\
			&= 12(3t^4)^3 12t^3 \\
			&= \boxed{3888 t^{15}}
		\end{align*}
	\end{solution}
	\item \begin{solution}
		\begin{align*}
			(f\circ f)'(u) &= f'(f(u)) f'(u) \\
			&= \boxed{\sin (\cos u) \sin u}
		\end{align*}
	\end{solution}
\end{enumerate}

\begin{problem}[4.]
	Given that $f$ and $g$ are both differential functions find the derivative of
	the following
	\be
		\item $\left((f\cdot g)\circ f\right)(t)$
		\item $\left(f \circ \left(\frac{f}{g}\right)\right)(x)$
	\ee
\end{problem}

\begin{enumerate}
	\item \begin{solution}
		\begin{align*}
			\frac{d}{dt}[((f\cdot g) \circ f)(t)] &= \frac{d}{dt}[f(f(t))\cdot g(f(t)) \\
			&= \frac{d}{dt}[f(f(t))]g(f(t)) + \frac{d}{dt}[g(f(t))]f(f(t)) \\
			&= \boxed{f'(f(t)) f'(t) g(f(t)) + g'(f(t)) f'(t) f(f(t))}
		\end{align*}
	\end{solution}
	\item \begin{solution}
		\begin{align*}
			\frac{d}{dx}\left[\left(f\circ \left(\frac{f}{g}\right)\right)(x)\right] &=
				\frac{d}{dx}\left[f\left(\frac{f(x)}{g(x)}\right)\right] \\
			&= f'\left(\frac{f(x)}{g(x)}\right)
				\frac{d}{dx}\left[\frac{f(x)}{g(x)}\right] \\
			&= \boxed{f'\left(\frac{f(x)}{g(x)}\right)\left[\frac{f'(x)g(x) -
				g'(x)f(x)}{[g(x)]^2}\right]}
		\end{align*}
	\end{solution}
\end{enumerate}

\end{document}
